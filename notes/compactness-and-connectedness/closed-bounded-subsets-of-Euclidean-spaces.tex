\documentclass[onecolumn]{ctexart}
\usepackage[utf8]{inputenc}
\usepackage{amsmath}
\usepackage{amssymb}
\usepackage{amsthm}
\usepackage{mathtools}
\usepackage{geometry}
\usepackage{graphicx}
\usepackage{float}
\usepackage{xcolor}
\usepackage{listings}
\usepackage{indentfirst}
\usepackage{bm}
\usepackage{tikz}
\usetikzlibrary{shapes,arrows}
\geometry{a4paper,scale=0.8}

\newcommand*\textinmath[1]{\thickspace\textnormal{#1}\thickspace}
\newcommand*\closure[1]{\overline{#1}}
\newcommand*\interior[1]{\mathring{#1}}

\newtheorem{definition}{Definition}
\newtheorem{theorem}{Theorem}
\newtheorem{proposition}{Proposition}
\newtheorem{lemma}{Lemma}
\newtheorem{corollary}{Corollary}
\newtheorem{remark}{Remark}
\newtheorem{example}{Example}

\title{Notes of "Closed Bounded Subsets of $\mathbb{E}^n$"}
\author{Jinxin Wang}
\date{}

\begin{document}

\maketitle

\section{Overview}
\begin{itemize}
  \item Def: Bounded subset of a Euclidean space
  \item Rmk: The closed and bounded subsets in $\mathbb{E}^n$ are of special interest to us because of their presence
  \item Def: Open cover and subcover of a topological space
  \item Examples of open covers and subcovers of topological spaces
  \begin{itemize}
    \item Eg: An open cover of $\mathbb{E}^1$: $\left\{ (k - 1, k + 1) \mid k \in \mathbb{Z} \right\}$ and its subcovers
    \item Eg: An open cover of $\mathbb{E}^2$: $\left\{ \interior{B}((x, 1)) \mid x = (a, b) \textinmath{in which} a, b \in \mathbb{Z} \right\}$ and its subcovers
    \item Eg: An open cover of $\lbrack 0, 1 \rbrack$: $\lbrack 0, \frac{1}{5} \rparen$, $\lparen \frac{4}{5}, 1 \rbrack$, $\left\{ (\frac{1}{n}, 1 - \frac{1}{n}) \mid n \in \mathbb{N} \wedge n > 2 \right\}$ and its subcovers
    \item Eg: An open cover of $\lparen 0, 1 \rbrack$: $\left\{ \lparen \frac{1}{n}, 1 \rbrack \mid n \in \mathbb{N} \wedge n > 1 \right\}$ and its subcovers
  \end{itemize}
  \item Thm: A necessary and sufficient condition for a subset of $\mathbb{E}^n$ is closed and bounded in terms of open covers
  \item Def: Compact topological space
  \item Rmk: Compactness is a topological property of a space
\end{itemize}

\section{Content}
\begin{definition}[Bounded subset of a Euclidean space]
  A subset $X$ of $\mathbb{E}^n$ is bounded if there exists a ball $B(O, r)$ in 
  which $O$ is the origin and $r > 0$ such that $X \subset B(O, r)$.
\end{definition}

\begin{definition}[Open cover and subcover of a topological space]
  Let $X$ be a topological space. A family $\mathfrak{F}$ of open subsets of $X$ 
  is called an open cover of $X$ if the union of the family is $X$. If 
  $\mathfrak{F}'$ is a subfamily of $\mathfrak{F}$ and $\bigcup \mathfrak{F}' = 
  X$, then $\mathfrak{F}'$ is called a subcover of $\mathfrak{F}$.
\end{definition}

\begin{example}[An open cover of $\mathbb{E}^1$: $\left\{ (k - 1, k + 1) \mid k \in \mathbb{Z} \right\}$ and its subcovers]
  The family of open subsets of $\mathbb{E}^1$ $\mathfrak{F} = \left\{ (k - 1, 
  k + 1) \mid k \in \mathbb{Z} \right\}$ is clearly an open cover of $\mathbb{E}^1$.

  Notice that $\mathbb{E}^1$ is unbounded, and $\mathfrak{F}$ does not have any 
  subcover. Suppose we remove the open interval centred at $k (k \in \mathbb{Z})$ 
  in $\mathfrak{F}$, then the point $x = k$ is not covered by any other open 
  intervals in $\mathfrak{F}$ since the distance between it and other integer 
  point in $\mathbb{E}^1$ is greater than or equal to $1$.
\end{example}

\begin{example}[An open cover of $\mathbb{E}^2$: $\left\{ \interior{B}((x, 1)) \mid x = (a, b) \textinmath{in which} a, b \in \mathbb{Z} \right\}$ and its subcovers]
  Analogous to the previous example, the family of open subsets of $\mathbb{E}^2$ 
  $\mathfrak{F} = \left\{ \interior{B}((x, 1)) \mid x = (a, b) \textinmath{in which} 
  \right. \\ \left. a, b \in \mathbb{Z} \right\}$ is an open cover of $\mathbb{E}^2$, but it does 
  not have any subcover since removing any open subset in $\mathfrak{F}$ will 
  cause the centre of the removed open subset to be uncovered.
\end{example}

\begin{example}[An open cover of $\lbrack 0, 1 \rbrack$: $\lbrack 0, \frac{1}{5} \rparen$, $\lparen \frac{4}{5}, 1 \rbrack$, $\left\{ (\frac{1}{n}, 1 - \frac{1}{n}) \mid n \in \mathbb{N} \wedge n > 2 \right\}$ and its subcovers]
  Let $\lbrack 0, 1 \rbrack$ have the subspace topology of $\mathbb{E}^1$. The 
  family of open sets $\mathfrak{F} = \left\{ \lbrack 0, \frac{1}{5} \rparen, 
  \lparen \frac{4}{5}, 1 \rbrack \right\} \cup \left\{ (\frac{1}{n}, 1 - \frac{1}{n}) 
  \mid \right. \\ \left. n \in \mathbb{N} \wedge n > 2 \right\}$ is an open cover of $\left[ 0, 1 
  \right]$.

  Notice that $\lbrack 0, 1 \rbrack$ is a closed and bounded subset of $\mathbb{E}^1$, 
  and $\mathfrak{F}$ has a finite subcover $\mathfrak{F}' = \lbrace \lbrack 0, 
  \frac{1}{5} \rparen, \\ \lparen \frac{4}{5}, 1 \rbrack, (\frac{1}{3}, \frac{2}{3}), 
  (\frac{1}{4}, \frac{3}{4}), (\frac{1}{5}, \frac{4}{5}), (\frac{1}{6}, \frac{5}{6}) 
  \rbrace$.
\end{example}

\begin{example}[An open cover of $\lparen 0, 1 \rbrack$: $\left\{ \lparen \frac{1}{n}, 1 \rbrack \mid n \in \mathbb{N} \wedge n > 1 \right\}$ and its subcovers]
  Let $\lparen 0, 1 \rbrack$ have the subspace topology of $\mathbb{E}^1$. The family of open sets $\mathfrak{F} = \left\{ \lparen \frac{1}{n}, 1 \rbrack \mid n \in \mathbb{N} \wedge n > 1 \right\}$ is an open cover of $\lparen 0, 1 \rbrack$.

  Notice that $\lparen 0, 1 \rbrack$ is not a closed subset of $\mathbb{E}^1$, 
  and $\mathfrak{F}$ does not have a finite subcover. Suppose that there is a 
  finite subcover $\mathfrak{F}'$ of $\mathfrak{F}$, then $\mathfrak{F}'$ consists 
  of finite open intervals with the form $\lparen \frac{1}{n}, 1 \rbrack$. Since 
  they are finite we can pick the open interval with the largest $n$, denoted by 
  $n_0$. Then $(0, \frac{1}{n_0})$ is not covered by $\mathfrak{F}'$, which 
  contradicts that $\mathfrak{F}$ is a subcover. Hence $\mathfrak{F}$ does not 
  have any finite subcover.
\end{example}

\begin{definition}[Compact topological space]
  A topological space $X$ is compact if every open cover of $X$ has a finite 
  subcover.
\end{definition}

\end{document}